\documentclass[12pt]{exam} % [answers] enables/disables answers
\usepackage{graphicx,amsmath,amssymb,subcaption,url,xspace}
% http://tex.stackexchange.com/questions/2291/how-do-i-change-the-enumerate-list-format-to-use-letters-instead-of-the-defaul#comment3172_2294
\usepackage[shortlabels]{enumitem}
% http://stackoverflow.com/questions/4768694/word-wrap-in-verbatim-environments#comment5279194_4769018
% \usepackage{listings}
% \lstset{breaklines=true}
% http://tex.stackexchange.com/questions/171803/change-font-size-of-the-verbatim-environment
\usepackage{fancyvrb}
\usepackage[acronym]{glossaries}
\ifprintanswers
\usepackage[]{todonotes} % insert [disable] to disable all notes.
\else
\usepackage[disable]{todonotes} % insert [disable] to disable all notes.
\fi
\usepackage{chronology}
% Remove/Insert inline to change the positioning of notes.
% Remove/insert fancyline to add/remove arrows.
\newcommand{\note}[1]{\todo[author=Pushpendre,color=blue!40,size=\small,fancyline,inline]{#1}}
\newcommand{\eg}{e.g.,\xspace}
\newcommand{\bigeg}{E.g.,\xspace}
\newcommand{\etal}{\textit{et~al.\xspace}}
\newcommand{\etc}{etc.\@\xspace}
\newcommand{\ie}{i.e.,\xspace}
\newcommand{\bigie}{I.e.,\xspace}
\renewcommand{\cite}[1]{cite\{#1\}}
\newcommand{\citet}[1]{citet\{#1\}}
\ifprintanswers
\newcommand{\remove}[0]{}
\else
\newcommand{\remove}[1]{}
\fi
% The EnvFullwidth environment is broken
% http://tex.stackexchange.com/questions/89251/how-to-break-long-solutions-in-exam-class
\makeatletter
\def\SetTotalwidth{\advance\linewidth by \@totalleftmargin
\@totalleftmargin=0pt}
\makeatother

\newcommand{\answer}[1]{
\ifprintanswers
\SetTotalwidth
  \begin{solution}[0in]#1\end{solution}
\else \fi
}
\renewcommand{\solutiontitle}{\noindent\textbf{}\enspace}
\title{Experiments to run}
\author{Pushpendre}
\makeglossaries

\begin{document}
\maketitle
\begin{questions}
  \question{What is the purpose of this project?\label{q:purpose}}
  \answer{Our project has the following three goals.
    \begin{enumerate}
    \item An ``intelligent'' machine should have the ability to
      quickly update its knowledge representation as new knowledge is
      added to its collection of rules. There is evidence that humans
      are able to do this as well \cite{pending}. We want to create
      algorithms and knowledge representations that meet this
      requirement better than the state of the art. It could be said
      that we want to solve the problem of logical omniscience. Our
      emphasis in this work would be on both the space efficiency for
      the storage of representations and time efficiency for updating
      representations:
      \begin{enumerate}
      \item How to ``quickly'' update the representation of entities
        quickly in light of increasing evidence.
      \item How to maintain a compact representation of knowledge.
      \end{enumerate}
    \item An ``intelligent'' machine should have the ability to make
      educated guesses about statements that it cannot explicitly
      prove to be true or false and it should be able quickly make
      these guesses. We want our machine's inference algorithm to be
      fast and robust.
    \item We want to investigate the benefits of explicitly
      guaranteeing symmetry and transitivity in vector space based
      models for logical deduction.
    \end{enumerate} }

  \question{What is the importance of this work?\label{q:importance}}
  \answer{The earliest symbolic systems such as SHRDLU \cite{SHRDLU}
    could be ``told'' facts about the world and those systems could
    then ``memorize'' the facts and ``use'' those facts as known true
    propositions to restrict and guide their deductions and
    inferences. \citet{norvig2003aima} called it the ability to
    ``Tell'' and ``Ask'' a database. However these systems suffered
    with two problems when used on their own in real life:
    \begin{enumerate}
    \item It proved impossible to build symbolic systems that could
      generalize to broad domains. These systems did not naturally
      possess the knowledge required for inference and at the time
      ``machine learning'' was not advanced enough to glean that
      knowledge automatically.
    \item The syntactic deduction and inference mechanisms of such
      symbolic logical systems proved to be too slow in general and
      sufficiently fast special purpose algorithms that were also
      widely applicable to important problem could not be
      developed. Slow inference became a bottleneck.
    \item These systems were not designed to handle uncertainity in
      knowledge and inference.
    \end{enumerate} One solution to this problem was to restrict the
    domain of the AI systems. This method was used to create
    successful domain specific software such as
    \cite{feignbaumSpectrography,feignbaumMedicalDiagnosis}. These
    systems thrived by limiting their area of operations to those high
    impact areas where the knowledge required for success could be
    encoded manually. The next breakthroughs were achieved by learning
    the knowledge directly from data through ``machine
    learning''. ``Bayesian Graphical Models'' like ``Factor Graphs''
    and ``Bayes Nets'' and ``Conditional Random Fields'' in particular
    solved the problem of reasoning with uncertainty and learning
    paradigms like ``Margin Based Methods'' were leveraged to create
    margin based robust models that performed well under the task
    specific loss, for example
    \cite{ganchevM3N,StructuredSVM,CollinsStructuredPerceptron}. Unfortunately,
    ``Markov Logic Networks (MLNs)'' the natural adaptations of
    graphical models for performing logical inference in propositional
    and first order logic proved to be too slow for most tasks like
    reading comprehension for various reasons , with large grounding
    being the most insurmountable \cite{orenAI2MLN}. More recently the
    specific methodology of ``deep learning'' and ``representation
    learning'' has created systems that can perform a wide array of
    large multi-class and structured classification, regression and
    transduction tasks. Spectacular results have been achieved on both
    ``perceptual'' tasks like object classification\cite{imagenet} and
    speeech recognition\cite{hintonspeech} and even the supposedly
    higher reasoning oriented tasks like machine translation and
    question
    answering\cite{devlinetal,mohitaiyyerUMDsocherAMA,manyOthers}. The
    methodology of ``Tensor/Matrix Factorization'' which has been
    heavily pushed by Sameer Singh and Tim Rocktaschel amongst others
    through their papers on \cite{SameerAndTim} together with the
    methods of ``Deep Learning'' can be clubbed together as ``vector
    space models''\cite{baroni} which can be further generalized into
    ``representation learning with arbitrary objects'' a specific
    example of which is the work by \cite{gaussianEmbeddingsMcCallum}.
    These systems and methods are cutting edge in 2015. Most
    pertinently these methods have been used for Edge Type Prediction
    in Knowledge Graphs, under the name of knowledge base completion
    \cite{socherKbCompletion,percyQAasRegularization} and for Natural
    Logic Relation prediction \cite{bowman3Papers}. The paradigm under
    which this work has occurred is the batch training and prediction
    paradigm and it has established strong baselines. For example
    \cite{bowmanWordnetPaper} demonstrated that a model that assigns
    continous deep representations to ``Natural Logic Relations'' and
    vector representations to ``Entities'' one could be trained with
    data randomly sampled from the transitive closure of the edges in
    Wordnet to predict other edges in the transitive closure. The
    problem with Bowman's work in particular and this type of work in
    general can be summed up in the following two points:

    \begin{enumerate}
    \item It is assumed that the training and testing data is
      generated through random samples of the possible arcs. The
      training and testing data in real databases is a lot more skewed
      than that. Intelligent machines would have to generally predict
      the relations in an unseen part of the graph based only on the
      representations learnt to represent one part of the graph.
    \item No constraints of symmetry and transitivity are enforced.
    \item No consideration is given to the model's ability to be
      ``Told'' and ``Asked'' about the inferences to be drawn based on
      increments in the information.
    \item No consideration is given to the fact that during trainig we
      do not retain random access to the entire training data. More
      specifically in a large semantic graph we would not have access
      to the entire transitive closure of a skeletal graph because it
      would be too expensive to store and also as humans learn during
      their lifetime they tend to update their representations of the
      world efficiently and perhaps even lazily.
    \item The fact that ``Searching'' is an important method of
      inference is completely ignored in the previous work that uses
      these models. Since all representation learning based methods
      essentially induce a graph based metric space that is
      inframetric\cite{BookWithDefinitionOfInframetric} which does not
      even necessarily obey the triangle inequality or
      symmetry~\cite{McCallumGaussianEmbeddings} therefore searching
      becomes even more important.
    \end{enumerate}

    In this work we would build models that improve the state of the
    art by answering these concerns.  }

  \question{What contemporary work is most similar in its application
    area to your project?}

  \answer{
    \begin{enumerate}
    \item ``Neural/Deep'' or ``Tensor/Matrix Factorization'' based
      ``Vector Space'' models under the guise of ``memory network''
      have recently been applied to the task of responding
      instantaneously to input presented to a user. These models have
      a component that is reminiscent of ``Ask'' and ``Tell''.
    \item The problem of zero shot learning in particular and domain
      adaptation in general are the general areas that address the
      difference between distributions of training and testing data.
    \item Theoretical work by \cite{GuillaimeAndSameer} presents some
      recommendations along with theorerical justifications that show
      that matrix factorization under hinge loss and logistic loss can
      preserve the relational structure of transitivity with high
      fidelity.
    \item A large body of work under the term multi-relational
      learning that was produced by Antoine Bordes and pushed by Leon
      Bottou amongst others has some overlap in its goals with our
      goals. Multi-relational refers to the fact that entities in
      graph databases that contain ``knowledge'' about the world can
      have more than one relation between them. For example two
      celebrities in free base may be co-stars in a movie and also
      spouse. This work along with the work on learning to generalize
      from triples has superficial similarity with our work but in
      some sense that work was a steady state analysis.
    \item Encoding FOL formulas in neural nets with quantified meta
      variables. This work was presented in the paper
      \cite{FirstOrderLogicLearninginArtificialNeuralNetworks}.
    \item Recently a trend of ad-hoc fitting the representations to
      manually-curated datasets has also started.
    \end{enumerate}

    Based on this enumeration we can say that ``Searching'' for
    inference with ``vector space methods and the fact that we do not
    either observe or maintain an explicit transitive closure of a
    semantic graph have been ignored previously and we are completely
    original in bringing these up.  }

  \question{Are you going to encode all rule in FOL or only some
    special types of rules? \label{q:areyou}} \answer{While we would
    like to consider encoding all kinds of arbitrarily rules in FOL
    with nested quantifications we have stuck to a very particular
    type of rules that comes up in knowledge bases and natural logic
    based entailment. \cite{guhaModelTheory2015} claims that a ``a
    significant fraction of axioms'' in the common sense knowledge
    base Cyc \cite{lenat1990Cyc} are of the form
    $(\forall x P(x,A) \implies Q(x,B))$ where P, Q are predicates and
    A, B are constants. This axiom can be represented as a type of
    edge composition rule in a graph. If x -> A through an arc of type
    P and A -> B through some unspecified arc type then x ->
    B. Similarly the natural logic inference rules can be encoded as
    path compositions with types. See
    \url{www.cs.jhu.edu/~prastog3/2015/08/05/edge-prediction-in-semantic-graphs.html}
    for details and notation.  }

  \question{Give me references for why you think your problem
    formulation is going to cover a large portion of real world
    scenarios?}  \question{Why do you think that your specific
    restrictions are good restrictions? Shouldn't your theory be as
    general as possible?}  \answer{See answer to~\ref{q:areyou}. That
    answer shows that the class of rules I am cosndering is a
    generally useful class. That's my defense.}

  \question{What are the different ways of framing the problem?}
  \answer{Well this is too broad a question. Apart from the previous
    methods that I detailed in~\ref{q:importance} one remark I would
    like to make is that generally the distinction that people make is
    that where do we want to put the information? Do we want to learn
    representation that are associated to each word such that complex
    relations like entailement and hypernymy and hyponymy fall out
    gracefully or do we want to learn complex continous representation
    for this relations as well? Also people differ in their
    application~\cite{SameerTimHowTheyAreDifferent} (PENDING) and in
    the way they grow an ontology~\cite{snow} }

  \question{What is the draw of using this vector based
    representations?  Why not just use symbolic logic or MLNs?}
  \answer{Vector space models of logic have emerged recently as an
    effective tool for combating the computational inefficiency of
    purely syntactic proof theoretic deduction systems and also for
    overcoming the ``scarcity'' of data \cite{addcitation}. See answer
    to~\ref{q:importance} for details}

  \question{What are the different ways people have built vector
    models?  What did Socher do? Think of RNTN, Bilinear Forms, Neural
    Tensor Forms} \answer{}

  \question{Do you know what Sameer Singh did?}  \answer{}

  \question{What is your most direct inspiration?}  \answer{My most
    direct inspiration is the work by Bowman.}

  \question{What kind of mathematical structures are there? Why is
    encoding mathematical structures important?}  \question{All right
    what is this online business? Why is incorporating online learning
    and representation of rules important?}  \question{What is the
    main problem that you are solving?}  \answer{See answer
    to~\ref{q:purpose}~\ref{q:areyou} for answers to the above.}

\section{Specifics}
  \label{sec:specifics}
  \question{Tell me the specific experiments that you will run? What
    results do you hope to get?}

  \answer{ To wit our goal is of encoding omnicience.
    \begin{enumerate}
    \item Add rules incrementally.
    \item Construct models whose Transient/steady state performance
      for edge detection in \newglossaryentry{rg}{name={RG},
        long={Rule Generated Graphs},
        description={Rule Generated Graphs are special graphs that
          obey rules of compositionality on how the edges are
          generated in the graphs.},
        first={Rule Generated Graphs~(RG)}} \gls{rg}
      high. The measure of performance is 0-1 loss while predicting
      existence of edges and edge types in the transitive closures of
      graphs or with added information.
    \end{enumerate}

    Specifically I would do edge prediction in the following types of graphs:
    \begin{description}
    \item[Compositional, Monochromatic Graphs]
      \begin{itemize}
      \item Symmetric
        \begin{itemize}
        \item Multi-Graph
        \item Uni-Graph
        \end{itemize}
      \item Transitive
        \begin{itemize}
        \item Multi-Graph
        \item Uni-Graph
        \end{itemize}
      \end{itemize}
    \item[Compositional, Two Colored Graphs]
      \begin{itemize}
      \item Symmetric
        \begin{itemize}
        \item Multi-Graph
        \item Uni-Graph
        \end{itemize}
      \item Transitive
        \begin{itemize}
        \item Multi-Graph
        \item Uni-Graph
        \end{itemize}
      \end{itemize}
    \item[Compositional, Rainbow Graphs]
      \begin{itemize}
      \item Symmetric
        \begin{itemize}
        \item Multi-Graph
        \item Uni-Graph
        \end{itemize}
      \item Transitive
        \begin{itemize}
        \item Multi-Graph
        \item Uni-Graph
        \end{itemize}
      \end{itemize}
    \end{description}

    For each graph imagine that there is a ideal $I$ that contains the
    rules expressed in reality in the graph. and imagine that there is
    a set or rules $R$ that is used to generate the transitive closure
    of $I = \tilde{I}$. I would perform the following tasks:

    \begin{itemize}
    \item Assuming that a randomly sampled subset of $T \subset \tilde{I}$ is available
      as training data build a model to predict the edges in $T^C =
      \tilde{I} - T$.
    \item Assuming that a growing filtration of $I$ is made available
      to us. Let's say that the elements of the filtrations are $D_1
      \subset D_2 \subset D_3\ldots I$ and that the transitive
      closures of the filtrations are as follows
      $\tilde{D_1} \subset \tilde{D_2} \subset \tilde{D_3} \subset
      \tilde{I}$
      Then I would learn models that would have high performance in
      predicting edges in either:
      \begin{itemize}
      \item $I$ given $D_i$
      \item $\tilde{D_I}$ given $D_i$
      \end{itemize}
    \item Assuming that a growing filtration of $R$ is made available
      to us. That is to say that we did not know all the
      compositionality rules in the beginning and steadily and surely
      these rules are being told to ut.
      These filtration are different because these are
      filtrations in the rules that are generating the graphs.
      These have an exponential effect on the edges in the graph.
      Adding a single rule is tantamount to adding a large number of
      edges in the completed graph.
    \end{itemize}

We would run experiments with the following two approaches:
\begin{itemize}
\item I would enlist constraints of transitivity and symmetry for help.
\item I would enlist on-line learning for help. While doing on-line
  learning we can grow rulesets in two different ways:
  \begin{enumerate}
  \item The rulesets can be strict skeletal trees
  \item The rulesets can be mid way between skeletal tree $S$ that generate
    the transitive closure and the transitive closure $C$. An
    interpolation between them with DAG like cyclic edges.
    We can create them from wordnet and freebase and PPDB etc.
  \end{enumerate}
\end{itemize}

There can be two types of edge predictions that we might want to do
\begin{enumerate}
 \item Under the transitive closure is there a direct edges between the two entities.
\item  Is y reachable from x under the ruleset? \note{This is only
    different from previous task if the transitive closure is not
    computable even though the $y$ is reachable from $x$ but the
    rulesets and the tasks we have considered are only interested in
    some relation holding between the two nodes in a graph and that is
  explicitly represented through an edge between the two nodes. So its
  not clear if reachability without path existence is ever useful.}
\end{enumerate}

While doing inference there are two kinds of search procedures we can use:
That basically represent two kinds of assumptions we make during
inference.
\begin{enumerate}
\item Model can faithfully represent the transitive closure.
\item The model requires search for representing the transitive closure.
\end{enumerate}


One method for training the model is as follows:
We can vary the number of things sampled so that the number of edges samples is
\begin{enumerate}
\item O(E)
 \item Constant where the constant is varied from
   \begin{itemize}
   \item 1000
   \item 100
   \item  10
   \item  1
   \item  0
   \end{itemize}
 \end{enumerate}
}

\note{We note that during training based on search there might be a case for
doing search and then not updating the representations in case we can
already make the desired inferences. The only problem is that our
criteria of derired inference would just be that we can infer the
knowledge that this example is giving us, but what about the knowledge
for other inferences that we can glean from the example?}

\question{What are all the experiments that you plan to do, give me a
  list.}
\answer{
It is difficult to enumerate the list of experiments that I would
perform because the experiments are designed through a combination of
factors. Let's list the factors and their order. By multiplying the
orders of the factors we would be able to compute the total number of
experiments.
\begin{description}
\item [Chromaticity of Graphs]
  By chromaticity of Graphs we mean the number of edge types in a
  graph. Consider the example that \cite{guha} talked about where he
  said that the majority of information in Cyc could be summarized
  through rules that could be represented as the composition of edges
  of two types. The istype edge and the hasProperty Edge. and there is
  only one type of composition happening in that graph. Therefore
  these graphs can be said to be two colored graphs. Similarly natural
  logic can be said to be 7 colored or rainbow colored.
\item [Symmetry/Transitivity] The majority of rules can be summarized
  as either encoding symmetry of a relation or transitivity of a relation.
\item [Multi-Graph versus Uni-Graph] The graphs can either allow more
  than one edge between nodes or they might constrain the graphs to
  have only one type of edge.
\item[On-line versus Batch Input] We can have a randomly sampled
  subset of $\tilde{I}$ available to us and use that as training data
  and only test ourselves on predicting the edges we did not see
  during training. Or we might use expanding filtrations and use the
  complements of those filtrations as the test data.
\item [The Filtration Types] We can either give subset of edges as
  filtrations and that too from a skeletal tree input or from the
  transitive closure or we can give compositional rules in the
  filtrations along with the seeds.
\item [Emphasis of the experiments] The emphasis of the experiments
  could be on showing that adding the constraints of symmetry and
  transititivty explicitly helps improve performance or it can be that
  to see the performance of different models for on-line learning.
\item [Search versus No Search during inference] The inference methds
  can be augmented through search based strategies. In case we don;t
  use search then some way of maintaining the transitive closure would
  have to be devised.
  \begin{enumerate}
  \item Either an explicit datastructure representing
    the rules presented so far can be kept in memory
  \item an explicit data structure representing the transitive closure
    can be grown.
  \item Just a model with vector representations can be trained.
  \end{enumerate}
\end{description}
}

\question{You enumerated goals for your project in
  \ref{q:purpose}~\ref{q:importance}. Tell me exactly how each
  experiment would help you reach those goals?}
\answer{We said that an intelligent machine should have the following
  abilities:
  \begin{description}
  \item[Quickly update representations with increasing knowledge] Our
    experiments on on-line learning would help test how well and how
    quickly the vector based methods are able to learn when provided a
    single example of a new edge that might make a number of other
    edges possible. Therefore our on-line learning experiment would be
    useful for this task.
  \item [Compact representation of knowledge] If we are able to train
  our model to predict the edges in the transitive closure without
  explicitly maintaining a transitive closure then we would have
  essentially represented a graph with $O(E)$ edges with $O(V)$
  space. Therefore our representations would be compact.
\item [Making Educated guesses about unseen data] That is essentially
  our performance metric for all the tasks. We are checking that we
  are able to guess the edge labels for unseen edges. Which means that
  all our experiments help us achieve this goal.
\item [Investigate methods for, benefits of symmetry, transitivity]
  We would use both neural and tensor based models for doing this
  task. Also see section~\ref{sec:enforc-symm-trans} for details on how I would
  encode symmetry and transitivity.
  \end{description}
}

\question{What is the timeline for completing all these experiments?}
\answer{
  Tentatively the timeline in units of weeks is the following:
  \begin{enumerate}[A)]
  \item Run experiments for investigating the benefits of symmetry and
    transitivity on different types of
    graphs(chromatic,multi-uni,symmetric/transitive) for batch
    training of edge prediction.
  \item Run experiments for investigating the effect of different
    filtrations provided as input during training and testing.
  \item Run experiments for the importance of search during inference.
  \item Put final touches on these experiments.
  \end{enumerate}
  \begin{chronology}[1]{1}{8}{\textwidth}[10cm]
    \event{2}{\large{A}}
    \event{4}{\large{B}}
    \event{6}{\large{C}}
    \event{8}{\large{D}}
  \end{chronology}
}

\question{How do you encode transitivity in your vector space
    models?}
\answer{See section~\ref{sec:enforc-symm-trans}}

\question{What are the different ways of doing this online learning
    task? How will you make sure that you can represent
    min$(O(E), O(V^2))$ graph in O(V) space?}
\answer{Well if while learning I never consrtuct a transitive closure of
  the graph  and at all points of time I only represent the graph by
  vectors that represent nodes and use some fixed number of parameters
to parameterize a predicate function that tells me whether an edge
exists between two vertices or not then I have essentially reduced the
space requirement from $O(E)$ to $O(V)$.}

\question{What is your experimental setup? How would you actually
  add the edges in your representation of the graph? How do you plan
  on ensuring that errors don't propagate egregiously?}
\answer{}

\question{How do you grow the ontology organically?}
\answer{}

\question{How are you encoding this thing called omnicience?}
\answer{ If we model the knowledge about the world as graph then one
    important ability that humans demonstrate and that has practical
    applications is the ability to predict relations and connections
    in an unseen part of the graph based only on the representations
    learnt to represent one part of the graph.

    This is omniscience.  }

\question{Do you plans for handling larger rule sets? More forms of
  rules? How would you test whether you will need to add those rules
  sets or not?}
\answer{}

\question{What about the fact that we can basically refer to any
  conjunction of properties or a clause as a type and then assign
  arbitrary properties to the entities referred to by that clause?}
\answer{}

\question{What is the rationale of the experiment where you keep
  adding 1 edge at a time and then see how far up the ladder
  transitivity holds?}
\answer{}

\end{questions}

\remove{
\section{Enforcing Symmetry and Transitivity}
\label{sec:enforc-symm-trans}
%\begin{lstlisting}[basicstyle=\normalsize\ttfamily]
\begin{Verbatim}[fontsize=\small]
Sam Bowman's conducted experiments on manually constructed models [1]
to test the NN/NTN models abilities to learn inferential patterns in
Natural Logic. Each model was created as follows
a. Create 80 sets by drawing from 7 entities (randomly)
b. Based on these 80 sets, one can ask 6400 pair wise questions about
   which natural logic relation can hold between the two sets. It is
   interesting to note that the natlog relations are mutually exclusive.
c. Split the 6400 formulaes, each formula is a 3-tuple, (S1, S2, Relation)
   Note, Bowman only feeds positive experiences into the net. He doesn't
   need to feed in negative experiences but that type of knowledge may
   actually be important.
d. He also had a prop logic proof system, that he needed to weed out test
   instances that could not be proved from the training instances and the axioms.
   Actually looking at this shows that there is a very strong structure that the
   axioms introduce. (Rocktaschel/Sameer's methods?)
   * The work can be done simply through forward induction by starting from
     existing knowledge and enumerating all possible theorems in the language.
     No need for backward chaining.
e. After weeding out the system's he trained the system. I wonder if adagrad etc.
   are really needed or if basic cvxopt methods could be used?
f. Now the training was done by varying the models, drawing the sets many times,
   changing the embedding sizes etc.
   [TORCH] I could wrap "Torch objects" for the training and testing.
       Basically write some code in Torch, wrap it with
       cython, expose it as a python class/object.
       The separation is maintained using.
       https://github.com/soumith/torch-ship-binaries
g. The goal is to reproduce the accuracy that bowman et.al reported of
   |     | Provable         | UnProvable       |
   | NTN | 98.1% (SE 0.67%) | 87.7% (SE 3.59%) |
   | NN  | 84.8% (SE 0.84%) | 68.7% (SE 1.58%) |
h. After that I would optimize the representations subject to constraints of symmetry, reflexivity, transitivity
   | SNo | Name     | Symm | Refl | Tran | Remarks                                                              |
   |   1 | entails  | NO   |      | ✓    | A subtype entails supertypes, More attributes entail Less attributes |
   |   2 | rentails | NO   |      | ✓    | Reverse of entails                                                   |
   |   3 | equiv    | ✓    | ✓    | ✓    | x = y                                                                |
   |   4 | altern   | ✓    |      | ✓    | x ∩ y = Φ ∧ x ∪ y ≠ D                                                |
   |   5 | negation | ✓    |      |      | x ∩ y = Φ ∧ x ∪ y = D                                                |
   |   6 | cover    | ✓    |      |      | x ∩ y ≠ Φ ∧ x ∪ y = D                                                |
   |   7 | indep    | ✓    |      | ✓    | (else)                                                               | > 0
   X^T R y > 0
   x^T R y > 0 and y^T R z > 0 then x^T R z > 0
   R = uu^T
i. The method to enforce these constraints would be:
   # NOTE: It is not possible to enforce transitivity without symmetry in plain bilinear models.
   #  therefore plain bilinear is only useful for equiv, altern, indep.
   #  but not for entailment.
   # NOTE: The activations that a R_i b produce actually only have to compete against other {R_i}
   # The constraints in my mind make sure that the activation of (a R_i b = b R_i a)
   # but they don't say how I would perform against (b R_i a) for other {R_i}
   # NOTE: entailment/rentailment are almost anti-symmetric in their nature.
   # a entails b
   # ==> a) b rentails a
   # ==> b) b "does not entail" a
   | Model                    | Symm                                   | Refl | Tran                   |
   | NTN                      | Symmetric Mode-3-Slice                 |      | TODO                   |
   | NN                       | Independent sum of NN activations(ISN) |      | TODO                   |
   | Plain Bilinear (useless) | symmetric W                            |      | Rank 1, symmetric, PSD |
   | Plain Linear             | v1 equals v2                           |      |                        |
   | Tensor Factorization     | TODO                                   |      | TODO                   |
j. Model the relations as follows:
   # NOTE: Actually remember that what we really want is that
   # If a R_3 b > (a R_1 b)  (a R_2 b) ...
   # Then b R_3 a > (b R_1 a) (b R_2 a)
   # Also note that the entails and rentails relations are impossible to formulate
   # Since they need to be anti-symmetric with entailment.
   # ISN is independent sum of neural network activations : score(a, b) = W(a + b)
   # IDN is independent difference of neural network activations : score(a, b) = W(a - b)
   | SNo | Name      | Approach 1           | Approach 2 | Approach 3                      |
   |   1 | entails   | BilinForm/Impossible | NN/IDN     | NTN/AntiSymmetric Mode-3-slices |
   |   2 | rentails  | BilinForm/Impossible | NN/IDN     | NTN/AntiSymmetric Mode-3-slices |
   |   3 | equiv     | Rank 1, PSD          | NN/ISN     | NTN/Symmetric Mode-3-Slice      |
   |   4 | exclusion | Rank 1, PSD          | NN/ISN     | NTN/Symmetric Mode-3-Slice      |
   |   5 | negation  | Symmetric            | NN/ISN     | NTN/Symmetric Mode-3-Slice      |
   |   6 | cover     | Symmetric            | NN/ISN     | NTN/Symmetric Mode-3-Slice      |
   |   7 | indep     | Rank 1, PSD          | NN/ISN     | NTN/Symmetric Mode-3-Slice      |
k. I would hope to achieve that the numbers in the unprovable case would go up:
   |     | Provable            | UnProvable         |
   | NTN | >= 98.1% (SE 0.67%) | > 87.7% (SE 3.59%) |
   | NN  | >  84.8% (SE 0.84%) | > 68.7% (SE 1.58%) |
l. If the numbers don't go up by introducing these hard constraints. then I can revert to the
   full NN / NTN case and then one by one add constraints of being ISN / IDN to the relations
   For example:
   | Sno | Ap-2(Original) | Ap-2(Constrained) | Ap-2(Entailment) | Ap-2(Others) | Ap-2(Other-1) | Ap-2(Others-[2345]) |
   |   1 | NN             | IDN               | IDN              | NN           | NN            | NN                  |
   |   2 | NN             | IDN               | IDN              | NN           | NN            | NN                  |
   |   3 | NN             | ISN               | NN               | ISN          | ISN           | NN                  |
   |   4 | NN             | ISN               | NN               | ISN          | NN            | ISN                 |
   |   5 | NN             | ISN               | NN               | ISN          | NN            | NN                  |
   |   6 | NN             | ISN               | NN               | ISN          | NN            | NN                  |
   |   7 | NN             | ISN               | NN               | ISN          | NN            | NN                  |
m. In order to make sure that the changes I see are really important changes I would need to do experiments
   on large enough sets, but for now just do it on artificial data, later on use other hierarchies:
   # prf means performance
   |     | Ap-2(Original) | AP-2(Constrained) | Ap-2(Entailment) | Ap-2(Others) | Ap-2(Other-1) | Ap-2(Others-[2345]) |
   | Prf |           87.7 |                97 |               90 |           93 |            88 |                  89 |
n. Bowman extracted 3 types of relations from wordnet, and they actively discard
   equivalent/synonyms by choosing only 1 term out of a cluster of synonyms/equivalents,
   also they remove antonyms and anything else.
   | hypernym     | entails   |
   | hyponym      | rentails  |
   | coordination | exclusion |
   * They limit the size of the vocabulary and extract all of the instances
     of these three relations for single word nouns in WordNet
     that are hyponyms of the node organism.n.01.
   * They balance the distribution of the classes, by slightly downsampling
     instances of the coordinate relation,
   * They had a total of 36,772 relations among 3,217 terms.
   * Embeddings were fixed at 25 dimensions and were initialized randomly or
     using distributional vectors from GloVe.
   * The feature vector produced by the comparison layer was fixed at 80 dimensions.
   * Results were reported using crossvalidation, choosing a disjoint 10% test sample
     for each of five runs.
   Now on the first look their results look incredibly impressive
   Remember this is a 3 class classification problem with 3677 test instances
   and
   |                       <r> |                  |                   |
   |          Portion of Train | NN (Random init) | NTN (Random Init) |
   |---------------------------+------------------+-------------------|
   | (#Train=#Test) (3677) 11% |               91 |                91 |
   |               (11031) 33% |               95 |                95 |
   |              (33094) 100% |               99 |                99 |
   Infact these results are so good that one has to ask what good is it to try and
   improve these results?
o. One can use this to improve the SICK textual entailment challenge?
   For example bowman et. al.'s model reached 76.9% acc. perhaps by adding
   this prior one could improve performance on that task.
   To be honest, I didn't even know about the SICK entailment challenge till bowman
   used it and I don't know if this is a useful task or not? TODO
p. Why am I writing this paper? Really all I want to find out is whether enforcing
   this constraint is a useful guaranteed way to mimic the logical behavior of
   symmetry and transitivity? And also I want to compare different optimization
   techniques for doing so. I am not interested in the SICK challenge. In the end
   I would have a classifier that could predict entailment between two terms. This
   method could be compares with the two vector based distance measures produced
   by mccallum. His method is one of introducing anti-symmetry in the scoring function.
   Once I have those two measures then I'd be able to use these optimization methods to
   learn entailment between phrases. Also I'd be able to credibly talk about sentential
   logic and how to embed it. But not really. I mean this is just the natural logic
   task. The important part is the optimization procedure that incorporates symmetry
   and asymmetry of predicates.
   What other tasks is this type of prior useful for?
   * Well the problem is that I am not learning these constraints of symmetry, This is sort
     just (generic knowledge/a rule) that I am trying to use. Some body tells me that a
     particular relation type is symmetric or asymmetric and I want to learn it.
     The best way to utilize such a knowledge short of changing the representation of
     all the entities in a knowledge base is to just learn a special type of representation
     for the predicate.
   * This is basically learning with a very specific kind of generic knowledge.
   * This could be used in QA over KB or in embedding KB relationships or in updating the embeddings of a KB.
     So some experiments would have to be over freebase / multi-relational data.
     Related work was:
     - Low dimensional embeddings of logic
     - Reasoning with neural networks for KB completion
     - Injecting Logical Background Knowledge into Embeddings for Relation Extraction
     -
q. There are three entailment datasets that I can use
   1. How we blessed distributional semantic evaluation; Baroni
   2. Entailment above the word level in distributional semantics.; Baroni
   3. SICK; Semeval-2014 task 1: Evaluation of compositional distributional
      semantic models on full sentences through semantic relatedness and
      textual entailment. The entailment dataset by Ellie; Baroni.
   But really beyond the idea of phrasal entailment the important idea is of
[1] @proceedings{bowman2014learning,
    Author = {Bowman, Samuel R and Potts, Christopher and Manning, Christopher D},
    Booktitle = {Proceedings of the AAAI Spring Symposium on Knowledge Representation and Reasoning},
    Title = {Learning Distributed Word Representations for Natural Logic Reasoning},
    Year = {2015}}
[2] Bowman Model
    1. Say we have words w1, w2
    2. Find their embeddings e1, e2
       * Optionally transform these embeddings through a tanh layer to get good results.
         Bowman did it for the wordnet data. But it's just crap shoot.
    3. Feed the two into a comparison function. f(e1, e2, Relation)
       Here f(x) is either leaky-relu or tanh
          * f(x) = max(x, 0) + 0.01 min(x, 0).
          * f(x) = tanh(x)
       a. [NN Comparator]  Just uses a v_r.f(W_r[e1, e2] + b_r)
       b. [NTN Comparator] Just uses a v_r.f(e1 . T_r . e2 + W_r[e1, e2] + b_r) or v_r.(f(e1 . T_r . e2) + f(W_r[e1, e2] + b_r))
       c. Bilinear  e1^T W_R e2
       d. Linear    v1 . e1 + v2 . e2
    4. Then feed this into a softmax classifier
    5. Regularize with L2
    6. Train with AdaGrad
\end{Verbatim}
}
\newpage
\printglossaries

\bibliographystyle{plain}
\bibliography{references.bib}

\end{document}

%%% Local Variables:
%%% mode: latex
%%% TeX-master: t
%%% End:
